\begin{table}[H]
\centering
\renewcommand{\arraystretch}{.65}
\caption{Top 10 Occupations}
\vspace{5mm}
\begin{tabular}{llrrr}
\toprule
\textbf{Period} & \textbf{Occupation} & \textbf{Count} & \textbf{Percentage} & \textbf{Total Inventors} \\
\midrule
1617-1750 & Gentleman            &   132 &  13.52\% & 976 \\
          & Esquire              &    83 &   8.50\% & 976 \\
          & Merchant             &    48 &   4.92\% & 976 \\
          & Knight               &    27 &   2.77\% & 976 \\
          & Watchmaker           &     7 &   0.72\% & 976 \\
          & Chymist              &     6 &   0.61\% & 976 \\
          & Engineer             &     6 &   0.61\% & 976 \\
          & Weaver               &     6 &   0.61\% & 976 \\
          & Captain              &     5 &   0.51\% & 976 \\
          & Doctor In Physick    &     4 &   0.41\% & 976 \\\midrule

1751-1800 & Gentleman            &   205 &  11.86\% & 1,728 \\
          & Esquire              &    87 &   5.03\% & 1,728 \\
          & Merchant             &    78 &   4.51\% & 1,728 \\
          & Engineer             &    51 &   2.95\% & 1,728 \\
          & Watchmaker           &    25 &   1.45\% & 1,728 \\
          & Ironmonger           &    21 &   1.22\% & 1,728 \\
          & Surgeon              &    19 &   1.10\% & 1,728 \\
          & Chymist              &    19 &   1.10\% & 1,728 \\
          & Hosier               &    15 &   0.87\% & 1,728 \\
          & Optician             &    14 &   0.81\% & 1,728 \\\midrule

1801-1850 & Gentleman            & 1,206 &  12.97\% & 9,299 \\
          & Engineer             &   803 &   8.64\% & 9,299 \\
          & Merchant             &   448 &   4.82\% & 9,299 \\
          & Esquire              &   418 &   4.50\% & 9,299 \\
          & Civil Engineer       &   282 &   3.03\% & 9,299 \\
          & Manufacturer         &   247 &   2.66\% & 9,299 \\
          & Chemist              &   138 &   1.48\% & 9,299 \\
          & Machine Maker        &   107 &   1.15\% & 9,299 \\
          & Machinist            &    86 &   0.92\% & 9,299 \\
          & Mechanic             &    76 &   0.82\% & 9,299 \\\midrule

1851-1899 & Engineer             & 21,289 &  11.03\% & 193,079 \\
          & Manufacturer         & 12,374 &   6.41\% & 193,079 \\
          & Gentleman            & 9,869 &   5.11\% & 193,079 \\
          & Merchant             & 5,626 &   2.91\% & 193,079 \\
          & Civil Engineer       & 3,257 &   1.69\% & 193,079 \\
          & Mechanical Engineer  & 2,434 &   1.26\% & 193,079 \\
          & Machinist            & 2,164 &   1.12\% & 193,079 \\
          & Mechanic             & 1,795 &   0.93\% & 193,079 \\
          & Manager              & 1,757 &   0.91\% & 193,079 \\
          & Chemist              & 1,667 &   0.86\% & 193,079 \\
\bottomrule
\end{tabular}
\caption*{\textit{Note}: Table displays the top 10 most frequent occupations in each of four periods: 1617-1750, 1751-1800, 1801-1850, and 1851-1899. Occupations are standardized using a series of rules: we drop plural mentions of occupations, and group together common abbreviations for `gentleman' and `esquire'. We then take the modal occupation of inventors (or if there does not exist a mode for a particular inventor, we assign a random occupation). We display the count (number of inventors in that period with the given occupation), the percentage of patents associated with that occupation, and the total number of inventors in the period.}
\label{tab:occupations_total}
\end{table}